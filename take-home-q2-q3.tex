\documentclass{article}
\usepackage{lmodern}
\usepackage{tikz}
\usetikzlibrary{automata,positioning}
\usepackage{enumitem}
\usepackage{amsmath}

\title{Machine Learning End Term Exam}
\author{Ahmad Shayaan IMT2014004 \\ H Vijaya Sharvani IMT2014022 \\ Indu Ilanchezian IMT2014024}


\begin{document}
\maketitle
\pagenumbering{gobble}
\pagenumbering{arabic}

\section*{Question 2}

To derive the solution to the modified linear regression leads to the generalized form of ridge regression.
\\
\\
Solution:-
\\
\\
Given the attribute $x_i = \hat{x_i} + \epsilon_i $, where the $\hat{x_i}$ are the true measurements and $\epsilon_i$ is the zero mean vector with covariance matrix $\sigma^2 I$
\\
Modified loss function
\begin{equation*}
W^* = argmin_w E_\epsilon\sum_{i=1}^{n}(y_i - W^T(\hat{x_i} + \epsilon_i))^2
\end{equation*}
Where W is the transformation vector.
\begin{equation}
W^* = argming_W E_\epsilon || Y - (X+\epsilon)W||_2^2 \tag{1}
\end{equation}
\\
Where
\begin{equation*}
	Y = \begin{bmatrix}
	y_1\\y_2\\\vdots\\y_n 
	\end{bmatrix}
\end{equation*}

\begin{equation*}
	X = \begin{bmatrix}
	\hat{x}_1^T\\\hat{x}_2^T\\\vdots\\\hat{x}_n^T
	\end{bmatrix}
\end{equation*}

\begin{equation*}
	\epsilon = \begin{bmatrix}
		\epsilon_1^T\\\epsilon_2^T\\\vdots\\\epsilon_n^T
	\end{bmatrix}
\end{equation*}

Expanding right hand side of equation 1.
\begin{equation*}
 E_\epsilon || Y - (X+\epsilon)W||_2^2  = E_\epsilon \bigg[(Y - (X+\epsilon)W)^T(Y - (X+\epsilon)W)\bigg]
\end{equation*}
\begin{equation*}
= E_\epsilon \bigg[ Y^TY + W^T(X+\epsilon)^T(X+E) - 2W^T(X+E)^TY \bigg] \tag{2}
\end{equation*}
To minimize the equation we will differentiate eq 2 wrt W.
\begin{equation*}
	\frac{\partial E_\epsilon \bigg[ Y^TY + W^T(X+\epsilon)^T(X+\epsilon)W - 2W^T(X+E)^TY \bigg]}{\partial W} = 0
\end{equation*}
We know that $\frac{\partial E(f(x))}{\partial}$  = $E\frac{\partial f(x)}{\partial x}$.

\begin{equation*}
	E_\epsilon \bigg[ \frac{\partial Y^TY}{\partial W } + \frac{\partial W^T(X+\epsilon)^T(X+\epsilon)W}{\partial W} -  2\frac{\partial W^T(X+E)^TY}{\partial W}\bigg] = 0
\end{equation*}

\begin{equation*}
	E\epsilon\big[ 2(X+\epsilon)^T(X+\epsilon)W -2(X+\epsilon)^TY\big] = 0
\end{equation*}

\begin{equation*}
	2E_\epsilon \big[(X+\epsilon)^T(X+\epsilon)W \big] -2 E_\epsilon \big[(X+\epsilon)^TY\big] = 0
\end{equation*}

\begin{equation*}
	E_\epsilon\big[(X^TX+\epsilon^T\epsilon +2\epsilon^TX\big)W] = E_\epsilon\big[(X+\epsilon)^TY\big])
\end{equation*}
\begin{equation*}
	E_\epsilon(X^TXW)+ E_\epsilon(\epsilon^T\epsilon W) + 2E_\epsilon(\epsilon^TX W) = E_\epsilon(X^TY) + E_\epsilon(\epsilon^TY)
\end{equation*}
We know that E(AB) = E(A)E(B) if A and B are independent variables and $E_f(h(x)) = \int_{-\infty}^{\infty}h(x)f(x)dx$.

\begin{equation*}
	\sum_{i=1}^{n}X^TXWP(\epsilon_i) + E_\epsilon(\epsilon\epsilon^T)E_\epsilon(W) +2E_\epsilon(X)E_\epsilon(\epsilon)  = \sum_{i=1}^{n} X^TYP(\epsilon_i) + E_\epsilon(Y)E_\epsilon(\epsilon)
\end{equation*}
We know that the noise is a zero mean Gaussian noise therefore E($\epsilon$) = 0
\begin{equation*}
	(X^TX + \sigma^2I)W = X^TY
\end{equation*}
\begin{equation*}
	W = (X^TX + \sigma^2I)^{-1}X^TY
\end{equation*}

therefore the solution of the minimization is 
\begin{equation*}
		W^* = (X^TX + \sigma^2I)^{-1}X^TY
\end{equation*}
This solution is same as the solution for Ridge regression
\begin{equation*}
	W^* = (X^TX  + \lambda I)^{-1}X^TY
\end{equation*}

\section*{Question 3}

$VC(\mathcal{H})$ is the maximum cardinality of any set of instances that can be shattered by $\mathcal{H}$. We say that $\mathcal{H}$ shatters a set of points if and only if it can assign any possible labeling to those points.
\begin{enumerate}
    \item We should show that the VC dimension $d_\mathcal{H}$ of any finite hypothesis space $\mathcal{H}$ is at most $log_{2}\mathcal{H}$.
\\Proof:
\\For any set of distinct points S of size n, there are $2^n$ distinct ways of labeling those points. This means that for $\mathcal{H}$ to shatter S it must contain at least $2^n$ distinct hypotheses. This tells us that if the VC dimension of $\mathcal{H}$ is n then we must have $2^n$ hypotheses, i.e. $2^n \leq |\mathcal{H}| $ or equivalently that $n = VC(\mathcal{H}) \leq log_{2}|\mathcal{H}|$.
    \item Consider a domain with n binary features and binary class labels. Let $\mathcal{H}$ be the hypothesis space that contains all decision trees over those features that have depth no greater than d. (The depth of a decision tree is the depth of the deepest leaf node.)
\\Proof: 
\\First note that any tree in $\mathcal{H}$ can be represented by a tree of exactly depth d in $\mathcal{H}$. So we will restrict our attention to trees of exactly depth d. All of these trees have $2^d$ leaf nodes. Also note that there are a total of $2^n$ examples in our instance space, which gives us an immediate upper bound on the VC-dimension of $\mathcal{H}$, i.e. $VC(\mathcal{H}) \leq 2^n$.
\\To get a lower bound let S contain the set of all possible $2^n$
instances. Since we have that $d \geq n$ it is straightforward to create a tree of depth n with a leaf node for each example and furthermore we can label the leaf nodes in all possible ways. This shows that we can shatter the set S with $\mathcal{H}$, which implies that $VC(\mathcal{H}) \geq 2^n$. Combining the upper and lower bound tell us that $VC(\mathcal{H}) = 2^n$.
\\Therefore, given some $d > 1$, we showed that a tight bound(theta bound) is possible such that $VC(\mathcal{H}) = d$ i.e. $d = d_H$.

\end{enumerate}

\end{document}
